\chapter{Profile Approach}

\section{Method}
\label{sec:method}
Consider the problem of having a time series that yield the demand of some product with one entry per week and generating a representative forecast for this. The problem can be framed as a game between Nature and a Robot, where each make a decision. Nature's decision ($s_i\in \Omega_S$) is to picks the true demand and the Robot's decision ($U_i(D)\in \Omega_U$) is guided to minimize a cost function defined viz 
\begin{equation}
	C:\Omega_U\times \Omega_S \mapsto \mathbb{R} 
\end{equation}
where $\Omega_U$ denotes the action space of the Robot and $\Omega_S$ denotes the action space of Nature. The objective of the Robot is to minimize the expected cost across it's actions. Let
\begin{equation}
	G\equiv \sum_{i=1}^n C(U_i(D),S_i),
\end{equation}
then
\begin{equation}
	\mathbb{E}[G|D,I] = \int ds_1ds_2\dots ds_n \sum_{i=1}^nC(U_i(D),s_i)p(s_1,s_2,\dots s_n|D,I)
	\label{eq:pen}
\end{equation}
where $i\in$ "weeks in forecast horizon" cover the weeks in the forecast horizon and $I$ denotes the background information. The minimum cost across the Robot's actions is defined viz
\begin{equation}
	\frac{d}{dU_i(D)}\mathbb{E}[G|D,I]\bigg|_{U_i(D) = U_i^*(D)} \stackrel{!}{=}0.
	\label{eq:crit}
\end{equation}
where $U^*_i(D)$ is the optimal decision rule for the Robot. In this project, the cost function is taken to be
\begin{equation}
	C(U_i(D),s_i) = (U_i(D)-s_i)^2
	\label{eq:cost}
\end{equation} 
Combining \EQref{eq:pen}, \EQref{eq:cost} and \EQref{eq:crit} yields 
\begin{equation}
	U^*_i(D) = \int ds_1ds_2\dots ds_n s_ip(s_1,s_2,\dots s_n|D,I).
	\label{eq:decision_rule}
\end{equation}
\EQref{eq:decision_rule} informs that the optimal decisions for the Robot should be the expected decisions of Nature, which makes intuitive sense considering the cost function (equation \eqref{eq:cost}).

To apply equation \EQref{eq:decision_rule}, the probability distribution $p(s_1,s_2,\dots s_n|D,I)$ needs to be specified. To do this, a host of assumptions regarding the process generating the data must be made.

\subsection{Simple Poisson Assumption}
A fixed decision rule that does not require updating at a later point it preferred by the stakeholders.

\begin{axiom}[Static and Independent]
	\label{axiom:static}
	The demand ($s_i$) for a given week of the year belong to the same static, Poisson distribution with a distinct rate parameter for each week. 
\end{axiom}

\begin{example}
	According to axiom \ref{axiom:static}, the sales from each week for a given brand and sub category belong to the same, static distribution. E.g. data from week 5 from 2021, 2022,... belong to a static distribution, and data from week 4 from 2021, 2022,... belong to another static distribution.
\end{example}

Using axiom \ref{axiom:static} 
\begin{equation}
	\begin{split}
		p(s_1,s_2,\dots s_n|D,I) &= p(s_1|s_2,\dots s_n,D,I)p(s_2|\dots s_n,D,I)\dots p(s_n|D,I)\\
		& = \prod_{i=1}^np(s_i|D,I),
	\end{split}
\end{equation}
meaning the optimal decision rule for the Robot can be written
\begin{equation}
	\begin{split}
		U_i^*(D) &= \mathbb{E}[S_i|D,I] \\
		& = \int d s_i s_ip(s_i|D,I)
	\end{split}
\end{equation}
According to axiom \ref{axiom:static}
\begin{equation}
	\begin{split}
		p(s_i|D,I)&=\int d\lambda_i p(s_i,\lambda_i|D,I)\\
		&=\int d\lambda_i p(s_i|\lambda_i,D,I)p(\lambda_i|D,I)\\
		&=\int d\lambda_i p(s_i|\lambda_i,D,I)\frac{p(D|\lambda_i,I)p(\lambda_i|I)}{p(D|I)}\\
	\end{split}
\end{equation}
where 
\begin{equation}
	p(D|I) = \int d\lambda_i p(D|\lambda_i,I)p(\lambda_i|I)
\end{equation}
and
\begin{equation}
	\begin{split}
		p(s_i|\lambda_i,D,I) & = p(s_i|\lambda_i,I)\\
		& = \text{Poi}(s_i|\lambda_i).
	\end{split}
\end{equation}
Note that the rate parameter $\lambda_i$ is associated to the demand $s_i$, whereas the data contains measurements of sales $y_j$ for $j$ representing past weeks. Thus, in order to formulate the likelihood function, a relationship between sales $y$ and demand $s$ must be assumed. In general

\begin{definition}[Demand and Sales]
	\label{def:demand_sales}
	Let $m_{b,c,i}$ denote the stock of a given brand $b$ and sub category $c$ for week $i$, then
	\begin{equation}
		y_{b,c,i} = \begin{cases}
			s_{b,c,i} & \text{ if } s_{b,c,i}\leq m_{b,c,i}\\
			m_{b,c,i} & \text{ else }
		\end{cases}.
	\end{equation}
\end{definition}

\begin{axiom}[Never Out of Stock]
	\label{axiom:demand_equal_sales}
	For simplicity, products are assumed to be never of of stock, meaning (referring to definition \ref{def:demand_sales}) $y_{b,c,i} = s_{b,c,i}$.
\end{axiom}

Using axioms \ref{axiom:static} and \ref{axiom:demand_equal_sales}, the likelihood function can be written
\begin{equation}
	p(D|\lambda_i,I) = \prod_{j\in \text{week }i}\text{Poi}(y_j|\lambda_i),
\end{equation}
where only data from the corresponding week is used in accordance with axiom \ref{axiom:static}. An obvious choice of prior would be the conjugate Gamma distribution (which is also the distribution with maximum entropy) with parameters $\alpha,\beta$, meaning
\begin{equation}
	p(\lambda_i|I) = \text{Ga}(\lambda_i|\alpha,\beta)
\end{equation}
This means
\begin{equation}
	\frac{p(D|\lambda_i,I)p(\lambda_i|I)}{p(D|I)}= \text{Ga}(\lambda_i|\alpha+N\bar{y},\beta +n)
\end{equation}
where
\begin{equation}
	\bar{y} \equiv \frac{1}{n}\sum_{j \in \text{week }i} y_j.
\end{equation}
Then
\begin{equation}
	\begin{split}
		\mathbb{E}[s_i|D,I] &=\sum_{s_i} s_i\int d\lambda_i \text{Poi}(s_i|\lambda_i|)\text{Ga}(\lambda_i|\alpha+N\bar{y},\beta +n_i)\\
		&=\int d\lambda_i \lambda_i\text{Ga}(\lambda_i|\alpha+N\bar{y},\beta +n_i)\\
		& = \frac{\alpha+n_i\bar{y}_i}{\beta+n_i}
	\end{split}
\end{equation}
In the limit of $\alpha,\beta \rightarrow 0$
\begin{equation}
	\begin{split}
		U_i^*(D) &=\lim\limits_{\alpha,\beta \rightarrow 0}\mathbb{E}[S_i|D,I]\\
		& = \bar{y}_i,
	\end{split}
	\label{eq:decision_rule2}
\end{equation}
Thus, using axioms \ref{axiom:static} and \ref{axiom:demand_equal_sales}, a fixed decision rule yielding the forecasted demand for week $i$ in the future, represented by \EQref{eq:decision_rule2}, is be obtained.

\subsection{Advanced Poisson Assumption}

\begin{axiom}[Static with coupled rate parameter]
	\label{axiom:coupled_rate_parameter}
	The demand ($s_i$) for a given week of the year belong to the same static Poisson Distribution where the rate parameter is coupled between different weeks.
\end{axiom}

Using axiom \ref{axiom:coupled_rate_parameter}
\begin{equation}
	\begin{split}
		p(s_1,s_2,\dots s_n| D,I) &= \int d\theta p(s_1,s_2,\dots s_n,\theta|D,I)\\
		& = \int d\theta p(s_1,s_2,\dots s_n|\theta,D,I)p(\theta|D,I)
	\end{split}
\end{equation}
where
\begin{equation}
	\begin{split}
		p(s_1,s_2,\dots s_n|\theta,D,I) &= p(s_1,s_2,\dots s_n|\theta,I)\\
		& = \prod_{j\in \text{forecast horizon}}\text{Poi}(s_j|\lambda(\theta,j))
	\end{split}
\end{equation}
and
\begin{equation}
	p(\theta|D,I) = \frac{p(D|\theta,I)p(\theta|I)}{p(D|I)}
\end{equation}
\begin{equation}
	p(D|\theta,I) = \prod_{i\in \text{training data}} \text{Poi}(s_i|\lambda(\theta,i)).
\end{equation}
This means
\begin{equation}
	\begin{split}
		U^*_j(D) &= \int d\theta\int ds_1ds_2\dots ds_n s_j\text{Poi}(s_j|\lambda(\theta,j))p(\theta|D,I)\\
		& = \int d\theta \int ds_js_j \text{Poi}(s_j|\lambda(\theta,j))p(\theta|D,I)\\
		& = \int d\theta \lambda(\theta,j) p(\theta|D,I)\\
		& = \mathbb{E}[\lambda|D,I]
	\end{split}
	\label{eq:decision_rule3}
\end{equation}
where $\lambda$ is some model, in this case a Fourier series
\begin{equation}
	\lambda(\theta,j) = a_0+\sum_{l=1}^{M}\bigg(a_l\cos\frac{2\pi}{N}j+b_l\sin\frac{2\pi}{N}j\bigg)
\end{equation}
and $\theta=\{a,b\}$.

